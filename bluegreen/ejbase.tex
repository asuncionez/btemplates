\documentclass{beamer}
\usepackage{epsfig}
\usepackage{graphicx}
\usepackage{epsfig}
\usepackage{subfigure}
\usepackage{hyperref}
\usepackage[spanish,activeacute]{babel}
\usepackage{fancybox}
\usepackage{helvet} %este guta
\usepackage{multirow}
\usepackage{rotating}
\usepackage{relsize}

\hypersetup{
    pdffitwindow=true,      % page fit to window when opened
    pdftitle={My title},    % title
    pdfnewwindow=true,      % links in new window
    colorlinks=true,       % false: boxed links; true: colored links
    linkcolor=black,          % color of internal links
    citecolor=green,        % color of links to bibliography
    filecolor=magenta,      % color of file links
    urlcolor=cyan           % color of external links
}

\usetheme{Boadilla} 
\useoutertheme{infolines} %pone numero de p'agina
\setbeamertemplate{navigation symbols}{}
%\usecolortheme[RGB={205,173,0}]{structure} 
%\usecolortheme[RGB={154,144,144}]{structure} %gray

\definecolor{mypink}{rgb}{0.7,0.2,0.3}
\definecolor{mynavyblue}{rgb}{0.08,0.06,0.4}

%%%%%%%%%% pone el t'itulo de los frames al centro
%\setbeamercolor{frametitle}{fg=mypink}
\setbeamerfont{frametitle}{series=\bfseries}

%%%%%%%%%%%%%%%%%%%%%%%%%%%%%%%%%%%%%%%%%%%%%%%%%%%%
\title[]{Evaluaci\'{o}n de sistemas de recomendaci\'{o}n\\Avances}
\author[]{Grupo de servicios de recomendaci\'{o}n\\CENIDET}
\institute[Cenidet]{}
\date{Marzo 2, 2011}

\begin{document}

%%%%%%%%%%%% includeonly
%\includeonlyframes{ggplot2}
%%%%%%%%%%%%%%%%%%%%%%%%%%%
\begin{frame}
  \titlepage
%  \begin{figure}
%    \includegraphics[width=2cm,keepaspectratio]{theimg/logoinaoe}
%  \end{figure}
\end{frame}

%%%%%%%%%%%%%%%%%%%%%%%%%%%
\section*{Contenido}
\begin{frame}{Contenido}
  %\color{mynavyblue}\textbf{Texto}
  \tableofcontents%[pausesections]
\end{frame}

%%%%%%%%%%%%%%%%%%%%%%%%%%% SECTION 1 INTRODUCTION
\section{Introducci\'{o}n}
%%%%%%%%%%%%%%%%%%%%%%%%%%% motiva
\begin{frame}[label=antecedentes]
\frametitle{}
    \begin{block}{Knowledge Discovery in Databases}
    Extracci'on no trivial de informaci'on expl'icita, previamente  desconocida y potencialmente 'util a partir de datos~\cite{hahsler:GeyerSchulz2002d}
    \end{block}
\end{frame} 
%----------
\begin{frame}[label=motiva]
\frametitle{Proceso de KDD}
\begin{figure}[h]
    \centering
    \includegraphics[keepaspectratio,width=12.5cm]{theimg/kddl}
\end{figure}
\end{frame} 
%----------
\begin{frame}[label=recsyskdd]
\frametitle{Metodolog'ia de evaluaci'on como proceso de KDD}
    \begin{figure}[h]
    \centering
    \includegraphics[keepaspectratio,width=12.3cm]{theimg/kddrecsys}
    \end{figure}
    \vspace{-1.5cm}
    \begin{figure}[h]
    \flushleft
    \includegraphics[keepaspectratio,,width=5cm]{theimg/kddrecsys2}
    \end{figure}
\end{frame} 
%%%%%%%%%%%%%%%%%%%%%%%%%%% SECTION 2 Pre-procesamiento
\section{Pre-procesamiento}
\begin{frame}[label=prepro]
\frametitle{Pre-procesamiento}
\vspace{0.3cm}
\begin{columns}[T]
    \column{.5\textwidth}
    \begin{itemize}
    %\setlength{\itemsep}{-\parsep}
    \item Datos incorrectos, an'alisis incorrecto.
    \item Formato adecuado para an'alisis.
    \end{itemize}
        
    \column{.5\textwidth}   
    %\vspace{1cm}
    \begin{itemize}
    %\setlength{\itemsep}{-\parsep}
    \item Eliminar registros vac'ios.
    \item Eliminar caracteres no v'alidos.
    \item Eliminar datos no relevantes.
    \item Datos faltantes: \textquestiondown qu'e se les hace?
    \end{itemize}     
\end{columns}
\vspace{0.3cm}
\begin{figure}[h]
    \centering
    \includegraphics[keepaspectratio,width=5cm]{theimg/escoba}
    \end{figure}
\end{frame}
%%%%%%%%%%%%%%%%%%%%%%%%%%% SECTION 3 Análisis descriptivo
\section{An'alisis descriptivo}
\begin{frame}[label=objetivoda]
\frametitle{An'alisis descriptivo: objetivo}
\begin{itemize}
    \setlength\itemsep{0.7cm}
    \item Dar un primer acercamiento a los datos.
    \item Describir los datos mediante t'ecnicas estad'isticas y gr'aficas.
    \item Extraer caracter'isticas importantes para an'alisis posteriores.
    \item Encontrar errores y patrones en los datos.
    \end{itemize}   
\end{frame} 
%----------
\begin{frame}[label=da]
\frametitle{An'alisis descriptivo: mapa}
\begin{figure}[h]
    \centering
    \includegraphics[keepaspectratio,width=9.5cm]{theimg/descrip}
    \vspace{0.2cm}
    \color{mynavyblue}\begin{tiny}[Sayad, 2010]\end{tiny}
\end{figure}
\end{frame} 

%%%%%%%%%%%%%%%%%%%%%%%%%%% SECTION 4 Análisis de Bookcrossing
\section{An'alisis univariable de Bookcrossing}
%%%%%%%%%%%%%%%%%%%%%%%%%%% 
\begin{frame}[label=bookdata]
\frametitle{An'alisis descriptivo: Bookcrossing}
\begin{center}
\url{http://www.bookcrossing.com/}
\end{center}
\begin{enumerate}
     \setlength\itemsep{0.7cm}
    \item \textbf{Perfil de usuario}: ID, ubicaci'on y edad, 
    \item \textbf{Libros}: ISBN, t'itulo, autor, a\~{n}o, editor y URLs, 
    \item \textbf{Ratings}: ID, ISBN, rating($1$-$10$); si el rating es impl'icito, el atributo toma el valor de $0$.
\end{enumerate}
\end{frame}
%----------
\begin{frame}[label=bookprepro]
\frametitle{Pre-procesamiento Bookcrossing}
\begin{table}[ht]
    \vspace{0.5cm}
    \begin{center}
    \begin{small}
    \begin{tabular}{|r|r|r|r|r|r|r|r|r|} \hline
    \textbf{Archivo} & \textbf{Raw} &  \textbf{e}&  \textbf{long} &  \textbf{sh} &  \textbf{rmvd} &  \textbf{dirtych} & \% \textbf{r}\\ [0.5ex]\hline
    Usuarios & $278,859$ & $0$ & $159$ & $2$ & $161$ & $836,097$  & $99$ \\\hline
    Libros  & $271,379$ & $0$ & $21,367$ & $0$ & $21,367$ & $1,250,065$&  $92$ \\\hline
    Ratings  & $1,149,780$ & $0$ & $0$ & $0$ & $0$ & $3,449,343$ &  $100$ \\\hline
    \end{tabular}
    \end{small}
    %\caption{Resultados del preprocesamiento}
    \end{center}
\end{table}
\end{frame} 
%----------
\begin{frame}[label=conteo]
\frametitle{Conteo}
\begin{table}[ht]
\begin{center}
\begin{tabular}{|r|r|r|} \hline
\textbf{Raw} & \textbf{Nulos} & $0$-$244$ \\ [0.5ex]\hline
$278,698$ & $110,724$ & $167,974$ \\\hline
\end{tabular}
\caption{Conteo de valores del atributo \textit{Edad}. \textit{Raw} es el n'umero total de ejemplos, \textit{Nulos} es el n'umero de personas que omitieron su edad y \textbf{0-244} es el n'umero de personas que si la proporcionaron.}
\label{tab:conteo}
\end{center}
\end{table}

\vspace{-0.3cm}

\begin{figure}[h]
    \begin{center}
    \subfigure[Todas las edades]{
    \includegraphics[width=5cm,keepaspectratio]{theimg/allage_bar}
    }
    \hspace{0.01cm}
    \subfigure[Nulos y no nulos]{
    \includegraphics[width=5cm,keepaspectratio]{theimg/nullage_bar}
    }
    
    \caption{Edades de los usuarios: en (a) se muestra la distribuci'on de las edades introducidas por el usuario, en (b) se muestra la comparaci'on de quienes omitieron la edad (TRUE) y los que si la registraron (FALSE).}
    \label{fig:gruposedades}
    \end{center}
\end{figure}
\end{frame}
%----------
\begin{frame}[label=atipicos]
\frametitle{Valores at'ipicos}
\begin{columns}[T]
    \column{.5\textwidth}
    \begin{itemize}
    \setlength\itemsep{0.7cm}
    \item Se calculan los cuartiles: $25$,$50$,$75$.
    \item Valores at'ipicos: $\pm 3$ desviaciones est'andar.
    \item Resultados: edades a partir de los $75$ a\~{n}os.
    \end{itemize}
        
    \column{.5\textwidth}   
    \vspace{-0.4cm}
    \begin{figure}[h]
        \centering
        \includegraphics[keepaspectratio,width=6cm]{theimg/boxplot}
    \end{figure}  
\end{columns}
\end{frame}
%----------
\begin{frame}[label=edadesposibles]
\frametitle{Rangos de edades}
\begin{figure}[h]
    \begin{center}
    \subfigure[Edades sin agrupar]{
    \includegraphics[width=5cm,keepaspectratio]{theimg/age_bar}
    }
    %\hspace{0.1cm}
    \subfigure[Agrupando las edades]{
    \includegraphics[width=5cm,keepaspectratio]{theimg/agebin_bar}
    }    
    %\caption{Agrupando las edades}
    \label{fig:gruposedades}
    \end{center}
\end{figure}
\end{frame}
%----------
\begin{frame}[label=analisisedad]
\frametitle{Resultados: edad}
\vspace{-0.2cm}
Se pueden identificar tres grupos de usuario:
\vspace{0.5cm}
\begin{itemize}
    \setlength\itemsep{0.7cm}
    \item Los que omitieron la edad.
    \item Los que proporcionaron edades at'ipicas: error o intencional.
    \item Los que proporcionaron edades dentro del rango posible.
    \end{itemize}  

\vspace{0.5cm} 
\begin{block}{}
\textquestiondown C'omo influyen estas categor'ias de edad en los resultados del sistema?
\end{block}

\end{frame} 
%----------
\begin{frame}[label=mdedad]
\frametitle{Medidas descriptivas: edad}
\begin{table}[ht]
\centering
\begin{tiny}
\begin{tabular}{|r|r|r|r|r|r|r|r|r|r|r|r|} \hline
Datos & Media($\mu$) & Med & Moda & Q1 & Q2 & Q3 & Var($\sigma$) & DS($\sigma^2$) & Sk($\gamma_1$) & K($\gamma_2$)\\ [0.5ex]\hline
con at'ip. & 34.76 & 32 & 24 & 24 & 32 & 44 & 208.16 & 14.43 & 9.05 & 1.18\\\hline
sin at'ip. & 34.59 & 32 & 24 & 24 & 32 & 44 & 177.04 & 13.31 & 2.62 & 0.6\\\hline
\end{tabular}
\end{tiny}
\end{table}
\begin{columns}[T]
\column{.5\textwidth}
\begin{itemize}
\item Edad promedio: $34.76$.
\item Mediana: $32$.
\item La edad m'as frecuente: $24$.
\item La \textbf{asimetr'ia} es positiva (i.e.,$2.63$) lo que indica que la distribuci'on est'a concentrada a la izquierda.
La \textbf{curtosis} es tambi'en positiva (i.e.,$0.6$), lo que indica que la distribuci'on es mas \textit{apuntada} que la normal.
\item Desviaci'on est'andar: $13.31$. %El $68$\% de los usuarios tienen $34.59$ $\pm$ $13.31$ a\~{n}os (i.e., entre $47.9$ y $21.28$).

\end{itemize}

\column{.5\textwidth}
\begin{figure}[h]
    \includegraphics[width=6cm,keepaspectratio]{theimg/filtage_hist}
     %\caption{Media y desviaci'on est'andar}
\end{figure}
\end{columns}

\end{frame}
%%%%%%%%%%%%%%%%%%%%%%%%%%% SECTION 5 Conclusiones y trabajo en curso
\section{Conclusiones y trabajo en curso}
\begin{frame}[label=conclusiones1]
\frametitle{Conclusiones}
\begin{itemize}
    \setlength\itemsep{0.7cm}
    \item Se present'o una metodolog'ia para la evaluaci'on de sistemas de recomendaci'on con atributos contextuales. 
    \item La metodolog'ia se basa en un enfoque de \textit{KDD} que incluye las fases de:
    \begin{itemize}
    \setlength\itemsep{0.2cm}
    \item Preprocesamiento de datos.
    \item An'alisis descriptivo.
    \item An'alisis de m'etricas.
    \item Generaci'on de reglas para clasificaci'on de m'etricas.
    \item Evaluaci'on.
    \item An'alisis de atributos.
    \end{itemize}   
\end{itemize}
\end{frame}
%----------
\begin{frame}[label=conclusiones2]
\frametitle{Conclusiones}
\begin{itemize}
 \setlength\itemsep{0.7cm}
 \item Se present'o el avance en las fases de preprocesamiento de datos y an'alisis descriptivo consistente en:
    \begin{itemize}
    \vspace{0.3cm}
    \setlength\itemsep{0.5cm}
    \item Estrategia b'asica de preprocesamiento. 
    \item Selecci'on de medidas descriptivas b'asicas.
    \item Aplicaci'on del an'alisis univariable sobre un conjunto de datos de un sistema de recomendaci'on. 
    \end{itemize}
\end{itemize}
\end{frame}
%----------
\begin{frame}[label=tactual]
\frametitle{Trabajo en curso}
\begin{itemize}
 \setlength\itemsep{0.7cm}
 \item An'alisis bivariable.
 \item Representaci'on de atributos para an'alisis por fases.
\end{itemize}
\end{frame}

%----------
\begin{frame}[label=bivar]
\frametitle{Ejemplo correlaci'on}
\begin{figure}[h]
    \begin{center}
    \subfigure{
    \includegraphics[width=5.5cm,keepaspectratio]{theimg/agerating1}
    }
    \hspace{0.01cm}
    \subfigure{
    \includegraphics[width=5.5cm,keepaspectratio]{theimg/agerating2}
    }    
    \caption{Edad-Rating}
   \end{center}
\end{figure}
\end{frame}
%%%%%%%%%%%%%%%%%%%%%%%%%%%
\section*{}
\begin{frame}[label=last]
\frametitle{}

\begin{center}
Gracias
\end{center}

\end{frame}

%%%%%%%%%%%%%%%%%%%%%%%%%%%
\bibliographystyle{apalike}
\bibliography{evan}
%%%%%%%%%%%%%%%%%%%%%%%%%%%
\end{document}

%%%%%% ejemplos

%\begin{block}{Introduction to {\LaTeX}}
%"Beamer is a {\LaTeX}class for creating presentations
%that are held using a projector..."
%\end{block}

%\begin{example}{Introduction to {\LaTeX}}
%"Beamer is a {\LaTeX}class for creating presentations
%that are held using a projector..."
%\end{example}

%\begin{alertblock}{Introduction to {\LaTeX}}
%"Beamer is a {\LaTeX}class for creating presentations
%that are held using a projector..."
%\end{alertblock}

%\begin{columns}[t]
%\column{.5\textwidth}
%\begin{block}{Column 1 Header}
%Column 1 Body Text
%\end{block}
%\column{.5\textwidth}
%\begin{block}{Column 2 Header}
%Column 2 Body Text
%\end{block}
%\end{columns}

%\shadowbox{Sample Text}
%\fbox{Sample Text}
%\doublebox{Sample Text}
%\ovalbox{Sample Text}
%\Ovalbox{Sample Text}

%\section*{Proyecto postdoctoral} * para que no aparezca en el contenido
